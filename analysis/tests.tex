\documentclass{article}
\usepackage{pgfplots}
\usepackage{caption}
\usepackage{graphicx}
\usepackage{float}
\pgfplotsset{compat=1.18}

\title{Experimental Evaluation of Input Variables and Their Impact}
\author{}
\date{}

\begin{document}

\maketitle

% ----------------------------- PLOT 1 -----------------------------
\section*{1. Number of Disjunctions vs. Time}
% TODO: Add explanation here.

\begin{figure}[H]
\centering
\begin{tikzpicture}
\begin{axis}[
    xlabel={Number of Disjunctions},
    ylabel={Time (s)},
    title={Disjunctions vs. Time},
    xmin=0, xmax=16,
    ymin=0, ymax=10.5,
    grid=major,
    width=12cm,
    height=8cm,
    only marks
]
\addplot+[
    only marks,
    mark=*,
] table[
    col sep=comma,
    x=disjunctions,
    y=time
] {disjunctions_vs_time.csv};
\end{axis}
\end{tikzpicture}
\caption{Dotplot of number of disjunctions vs. time (seconds, capped at 10).}
\end{figure}

% ----------------------------- PLOT 2 -----------------------------
\section*{2. Number of Disjunctions vs. Number of States}
% TODO: Add explanation here.

\begin{figure}[H]
\centering
\begin{tikzpicture}
\begin{axis}[
    xlabel={Number of Disjunctions},
    ylabel={Number of States},
    title={Disjunctions vs. States},
    xmin=0, xmax=16,
    grid=major,
    width=12cm,
    height=8cm,
    only marks
]
\addplot+[
    only marks,
    mark=*,
] table[
    col sep=comma,
    x=disjunctions,
    y=states
] {disjunctions_vs_states.csv};
\end{axis}
\end{tikzpicture}
\caption{Dotplot of number of disjunctions vs. number of states (timeouts excluded).}
\end{figure}

% ----------------------------- PLOT 3 -----------------------------
\section*{3. Constant Size vs. Time}
% TODO: Add explanation here.

\begin{figure}[H]
\centering
\begin{tikzpicture}
\begin{axis}[
    xlabel={Constant Value},
    ylabel={Time (s)},
    title={Constant Size vs. Time},
    xmin=0, xmax=550,
    ymin=0, ymax=10.5,
    grid=major,
    width=12cm,
    height=8cm,
    only marks
]
\addplot+[
    only marks,
    mark=*,
] table[
    col sep=comma,
    x=constant,
    y=time
] {constant_vs_time.csv};
\end{axis}
\end{tikzpicture}
\caption{Dotplot of constant size vs. time (seconds, capped at 10).}
\end{figure}

% ----------------------------- PLOT 4 -----------------------------
\section*{4. Constant Size vs. Number of States}
% TODO: Add explanation here.

\begin{figure}[H]
\centering
\begin{tikzpicture}
\begin{axis}[
    xlabel={Constant Value},
    ylabel={Number of States},
    title={Constant Size vs. States},
    xmin=0, xmax=550,
    grid=major,
    width=12cm,
    height=8cm,
    only marks
]
\addplot+[
    only marks,
    mark=*,
] table[
    col sep=comma,
    x=constant,
    y=states
] {constant_vs_states.csv};
\end{axis}
\end{tikzpicture}
\caption{Dotplot of constant size vs. number of states (timeouts excluded).}
\end{figure}

% ----------------------------- PLOT 5 -----------------------------
\section*{5. Quantifier Depth vs. Time}
% TODO: Add explanation here.

\begin{figure}[H]
\centering
\begin{tikzpicture}
\begin{axis}[
    xlabel={Quantifier Depth},
    ylabel={Time (s)},
    title={Quantifier Depth vs. Time},
    xmin=0, xmax=7,
    ymin=0, ymax=10.5,
    grid=major,
    width=12cm,
    height=8cm,
    only marks
]
\addplot+[
    only marks,
    mark=*,
] table[
    col sep=comma,
    x=depth,
    y=time
] {depth_vs_time.csv};
\end{axis}
\end{tikzpicture}
\caption{Dotplot of quantifier depth vs. time (seconds, capped at 10).}
\end{figure}

% ----------------------------- PLOT 6 -----------------------------
\section*{6. Quantifier Depth vs. Number of States}
% TODO: Add explanation here.

\begin{figure}[H]
\centering
\begin{tikzpicture}
\begin{axis}[
    xlabel={Quantifier Depth},
    ylabel={Number of States},
    title={Quantifier Depth vs. States},
    xmin=0, xmax=7,
    grid=major,
    width=12cm,
    height=8cm,
    only marks
]
\addplot+[
    only marks,
    mark=*,
] table[
    col sep=comma,
    x=depth,
    y=states
] {depth_vs_states.csv};
\end{axis}
\end{tikzpicture}
\caption{Dotplot of quantifier depth vs. number of states (timeouts excluded).}
\end{figure}

\end{document}